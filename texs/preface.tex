% preface
{\huge \textbf{Vorwort}}
\vspace{8mm}

Der vorliegende Text ist als Hilfestellung für neue Studienassitentinnen und
Studienassistenten der Fächergruppe Baumechanik 3 (früher Mechanik B3), Dynamik
VT und Mechanik-Dynamik (früher Mechanik 2) gedacht. Er soll in die allgemeine
Organisation der Lehrveranstaltung, die Handhabung der \TeX{}-Dokumente und in
die Bedienung des Teach Centers einführen.

Die Formelsammlung und Übungsblätter in ihrer jetzigen Form gehen auf
Jürgen Hackl %\footnotemark[1]
zurück. Sie sind zum Zeitpunkt des Entstehens
dieses Texts bereits mehr als fünf Jahre kontinuierlich weiterentwickelt worden
und gewachsen. Deshalb kann die Organisation für den, wahrscheinlich in \LaTeX{}
noch ungeübten, neuen Studienassistenten bzw. die neue Studienassistentin etwas
unübersichtlich sein.

Wir, Studienassistenten der ersten \glqq{}Generation\grqq{} nach Hackl, taten
uns zu Beginn unserer Tätigkeit am AM:BM relativ schwer, da wir zuvor weder
\LaTeX{} noch Corel Draw benutzt hatten. Mittlerweile haben wir jedoch
in Summe fünf Jahre Erfahrungen als Studienassistenten am AM:BM sammeln können,
und halten es für wichtig, diese in irgendeiner Form unseren Nachfolger 
zugänglich zu machen. Daraus ist die Idee entstanden, diesen einführenden Text
zu schreiben.

Die Dokumente zur Erstellung dieses Leitfades werden durch \twrite{git} 
versionskontrolliert. Das repository kann vom \twrite{github}-Server mit der
URL \url{\tagithub} gecloned werden. Falls sich einer unserer Nachfolger bzw.
Nachfolgerinnen wünscht, diese Arbeit zu vervollständigen - was uns natürlich 
sehr freuen würde - so kann eine \twrite{push}-Erlaubnis von einem von uns
erbeten werden.

Wir hoffen, dass wir mit diesem Text unser Ziel, nämlich dir den Einstieg in
deine Tätigkeit als Studienassistent einfacher zu gestalten, erreicht haben und
wünschen dir viel Erfolg und Spaß am AM:BM.

\begin{flushleft}
Graz, Juni 2015 
\end{flushleft}
\begin{flushright}
Dominik Pölz\\Daniel Schöllhammer
\end{flushright}

%\footnotetext[1]{Institit für Bau-und Infrastrukturmanagement, ETH Zürich,
%Stefano-Franscini-Platz 5, 8093 Zürich}