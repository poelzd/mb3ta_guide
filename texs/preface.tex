% preface

{\huge \textbf{Vorwort}}\\
\vspace{8mm}

Der vorliegende Text ist als Hilfestellung für neue Studienassitentinnen und
Studienassistenten der Fächergruppe Baumechanik 3 (früher Mechanik B3), Dynamik
VT und Mechanik-Dynamik (früher Mechanik 2) gedacht. Er soll in die Handhabung
der Formelsammlung, der Übungsblätter und der allgemeinen Organistion der 
Lehrveranstaltungen einführen.

Die Formelsammlung und Übungsblätter in ihrer jetzigen Form gehen auf
Jürgen Hackl %\footnotemark[1]
zurück. Sie sind zum Zeitpunkt des Entstehens
dieses Texts bereits mehr als 5 Jahre kontinuierlich weiterentwickelt worden und
gewachsen. Deshalb kann die Organisation für die, wahrscheinlich in \LaTeX{}
noch ungeübte, neue Studienassistentin (bzw. den Studienassistenten) etwas 
unübersichtlich sein.

Wir, Studienassistenten der ersten \glqq{}Generation\grqq{} nach Hackl, taten
uns zu Beginn relativ schwer, da wir weder Erfahrungen in \LaTeX{} noch Corel
Draw hatten und haben relativ viel Zeit aufgebracht, um uns einzuarbeiten.
Um die Situation für unsere Nachfolger etwas einfacher zu gestalten haben wir
uns dazu entschlossen, diesen kurzen Leitfaden zu schreiben.

Wir hoffen, dass wir mit diesem Text unser Ziel, nämlich dir den Einstieg
einfacher zu machen, erreicht haben und wünschen dir viel Erolg und viel Spaß
bei deiner Tätigkeit als Studienassistent am AM:BM.

\begin{flushleft}
Graz, Juni 2015 
\end{flushleft}
\begin{flushright}
Dominik Pölz\\Daniel Schöllhammer
\end{flushright}

%\footnotetext[1]{Institit für Bau-und Infrastrukturmanagement, ETH Zürich,
%Stefano-Franscini-Platz 5, 8093 Zürich}