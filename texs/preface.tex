% preface
{\huge \textbf{Vorwort}}\\
\vspace{8mm}

Der vorliegende Text ist als Hilfestellung für neue Studienassitentinnen und
Studienassistenten der Fächergruppe Baumechanik 3 (früher Mechanik B3), Dynamik
VT und Mechanik-Dynamik (früher Mechanik 2) gedacht. Er soll in die Handhabung
der Formelsammlung, der Übungsblätter und der allgemeinen Organistion der 
Lehrveranstaltungen einführen.

Die Formelsammlung und Übungsblätter in ihrer jetzigen Form gehen auf
Jürgen Hackl %\footnotemark[1]
zurück. Sie sind zum Zeitpunkt des Entstehens
dieses Texts bereits mehr als 5 Jahre kontinuierlich weiterentwickelt worden und
gewachsen. Deshalb kann die Organisation für die, wahrscheinlich in \LaTeX{}
noch ungeübte, neue Studienassistentin (bzw. den Studienassistenten) etwas 
unübersichtlich sein.

Wir, Studienassistenten der ersten \glqq{}Generation\grqq{} nach Hackl, taten
uns zu Beginn unserer Tätigkeit am AM:BM wirklich schwer, da wir weder 
Erfahrungen in \LaTeX{} noch Corel Draw hatten. Mittlerweile haben wir jedoch
zusammengezählt 5 Jahre Erfahrungen als am AM:BM gesammelt, und halten es für
wichtig, dieses Know-How für unsere Nachfolger in irgendeiner Form zu erhalten.
Um euch den Start in die Studienassistenz zu erleichtern, haben wir also diesen
Leitfaden verfasst.

Die Dokumente zur Erstellung dieses Leitfades werden durch \twrite{git} 
versionskontrolliert. Das repository kann vom \twrite{github}-Server mit der
URL \url{\tagithub} gecloned werden. Falls sich einer unserer Nachfolger bzw.
Nachfolgerinnen wünscht, diese Arbeit weiterzuentwicklen (was uns natürlich 
sehr freuen würde), so kann eine \twrite{push}-Erlaubnis von einem von uns
erbeten werden.

Wir hoffen, dass wir mit diesem Text unser Ziel, nämlich dir den Einstieg
einfacher zu gestalten, erreicht haben und wünschen dir viel Erolg und Spaß bei
deiner Tätigkeit als Studienassistent am AM:BM!

\begin{flushleft}
Graz, Juni 2015 
\end{flushleft}
\begin{flushright}
Dominik Pölz\\Daniel Schöllhammer
\end{flushright}

%\footnotetext[1]{Institit für Bau-und Infrastrukturmanagement, ETH Zürich,
%Stefano-Franscini-Platz 5, 8093 Zürich}