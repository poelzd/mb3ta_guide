% chapter 1 : structure
\chapter{Struktur}

In diesem Kapitel wird die Ordner- und Datenstruktur erläutert, die dem Skriptum
und der Übungssammlung zugrunde liegt.

\section{\"{U}bersicht}

Alle relevanten Dateien befinden sich im Laufwerk \twrite{N:}
(\twrite{MECH/BS1/}), wie in \figref{fig:uebersicht} dargestellt wird.
Im Folgenden findet sich noch eine kurze Erklärung zu den einzelnen Ordnern.

% tikzpic from [drive N:] to [Lehrveranstaltungen]
\begin{figure}[h!]
\begin{tikzpicture}[node distance = 1.5cm, auto]
  % 1st level nodes
  \node [root] (drive) {Laufwerk N:};
  \node [root, below of=drive] (parent) {Bauer-StudienassistentInnen};
  \node [root, right of=drive, node distance=4.5cm] (public) {Public};
  \node [obsolete, right of=public, node distance=3.5cm] (unpublic)
   {unwichtig};
  \node [virtualnode, below of=parent] (vn) {};
  \node [sdir, right of=vn, node distance = 3cm] (admin) 
   {Administratives};
  \node [sdir, below of=admin] (bauer) {Bauer};
  \node [sdir, below of=bauer] (lvs) {\bf Lehrveranstaltungen};
  \node [sdir, below of=lvs] (sicherung) {Sicherung Archiv};
  \node [sdir, below of=sicherung] (sonst) {Sonstiges};
  \node [sdir, below of=sonst] (studis) {StudienassistentInnen-Privat};
  \node [sdir, below of=studis] (tc) {Teach Center};
  \node [sdir, below of=tc] (ulz) {ULZ};
  % 1st level edges
  \path [line] (drive) -- (public);
  \path [line] (public) -- (unpublic);
  \path [line] (drive) -- (parent);
  \path [line] (drive) -- (parent);
  \path [line] (parent) |- (admin);
  \path [line] (parent) |- (bauer);
  \path [line] (parent) |- (lvs);
  \path [line] (parent) |- (sicherung);
  \path [line] (parent) |- (sonst);
  \path [line] (parent) |- (studis);
  \path [line] (parent) |- (tc);
  \path [line] (parent) |- (ulz);
  %
  % 2nd level nodes
  \node [description, right of=admin, node distance=5cm] (stunden)
   {Studenliste, Studienassistenz};
  \node [obsolete, right of=bauer, node distance=5cm] (unbauer)
   {unwichtig};
  \node [description, right of=lvs, node distance=5cm] (wichtiglvs)
   {Genaueres\\ im Folgenden};
  \node [obsolete, right of=sicherung, node distance=5cm] (unsicherung)
   {unwichtig};
  \node [obsolete, right of=sonst, node distance=5cm] (unsonst)
   {unwichtig};
  \node [description, right of=studis, node distance=5cm] (ordner)
   {Eigener Ordner, Diverses};
  \node [description, right of=tc, node distance=5cm] (wichtigtc)
   {Guide zum TC,\\ Alte Daten};
  \node [obsolete, right of=ulz, node distance=5cm] (unulz)
   {unwichtig};
  % 2nd level edges
  \path [line] (admin) -- (stunden);
  \path [line] (bauer) -- (unbauer);
  \path [line] (lvs) -- (wichtiglvs);
  \path [line] (sicherung) -- (unsicherung);
  \path [line] (sonst) -- (unsonst);
  \path [line] (studis) -- (ordner);
  \path [line] (tc) -- (wichtigtc);
  \path [line] (ulz) -- (unulz);
\end{tikzpicture}
\caption{Übersicht zur Struktur im Ordner \twrite{Bauer-StudienassistentInnen}.}
\label{fig:uebersicht}
\end{figure}

\begin{itemize}
  \item \twrite{Administratives:} Hier befinden sich einige verschiedene
    Dateien, primär {\tt .doc}-files. Die einzig wirklich relevanten sich jedoch
    die {\it Stundenliste} und die Liste der {\it Studienassistenz}. Erstere ist
    eine Vorlage, um darin die geleisteten Stunden eintragen zu können. Letztere
    beinhaltet die Kontaktdaten aller Studienassitenten.
  \item \twrite{Lehrveranstaltungen:} Der wichtigste Ordner hier, wird
    nachfolgend noch genauer erklärt.
  \item \twrite{StudienassistentInnen-Privat:} Jede Studienassistentin bzw.
    jeder Studienassistent kann sich hier einen eigenen Ordner anlegen.
    Falls man die Studenliste digitial führen möchte (was grundsätzlich zu
    empfehlen ist), kann diese hier gespeichert werden.
  \item \twrite{Teach Center:} Hier befindet sich eine 
    {\it Einführung ins Teach Center}, die jedoch etwas länglich ist, und viele
    Funktionen anspricht, die wir im Rahmen der LV nie benötigt haben
    (Umfragen, Forum, etc.). Weiters befinden sich veraltete Sicherungskopien
    der Daten im TC darin, die eigentlich nicht mehr benötigt werden. 
\end{itemize}

\section{Ordner \twrite{Lehrveranstaltungen}}

Wie man in \figref{fig:lvs} erkennen kann, befinden sich im Unterordner\\
\twrite{Lehrveranstaltungen/Unterlagen\_Dynamik/} alle wichtigen Dateien zu
den Fächern der Studienassistenten. Bei den beiden anderen Fächern
\twrite{Kontinuumsmechanik} und \twrite{Materialtheorie} müssen in der Regel
keine Änderungen vorgenommen werden. In untenstehender Auflistung wird kurz auf
die in der Abbildung vorkommenden Ordner eingegangen.

\begin{itemize}
  \item \twrite{Formelsammlung\_Dynamik:} In diesem Ordner befindet sich das
    Skriptum (welches nun als \glqq{}Vorlesungsbehelf\grqq{} bekannt ist, früher
    aber \glqq{}Formelsammlung\grqq{} hieß). Im Folgenden wird noch genauer
    darauf eingegangen.
  \item \twrite{\ddot{U}bungsbl\ddot{a}tter\_Dynamik:} Hier sind alle Dateien
    zu den Angaben der Übungen, der Lösungen und der durchgerechneten Beispiele.
    Etwas später wird ebenfalls noch genauer darauf eingegangen.
  \item \twrite{LV\_Organisation/Archiv:} In diesem Ordner sind primär die
    {\tt .pdf}-files der Formelsammlung und der Übungsbätter der
    Lehrveranstaltungen der vergangenen Jahre gespeichert. Am Ende des
    Sommersemesters sollten die Unterlagen der endenden Lehveranstaltung hier
    abgespeichert werden.
   \item \twrite{LV\_Organisation/Aktuelle\_Lehrveranstaltung:} Hier sind alle
     notwendigen Dateien der momentanen Lehrveranstaltung gespeichert. Zum
     einen sollen hier die Beurteilungslisten der Prüfungen im {\tt .xls}
     Format (also als {\tt Excel}-Dateien) abgelegt werden. Außerdem sind die
     Prüfungsmodalitäten als {\tt .pdf} abgespeichert. Zum Stundenplan sind
     sowohl die erzeugenden {\tt Excel}-sheets, als auch die exportierten
     {\tt .pdf}-Dateien enthalten. Außerdem gibt es {\tt .xls}-Listen zur 
     Übungsblattabgabe.
\end{itemize}

% tikzpic from [Lehrveranstaltungen] to [Formelsammlung]/[Uebung]
\begin{figure} 
\begin{tikzpicture}[node distance = 1.5cm, auto]
  % 1st level nodes
  \node [root] (parent) {Lehrveranstaltungen};
  \node [virtualnode, below of=parent] (vn) {};
  \node [ldir, right of=vn, node distance=4cm] (kmech)
   {Unterlagen\_Kontinuumsmechanik};
  \node [ldir, below of=kmech] (material) {Unterlagen\_Materialtheorie};
  \node [ldir, below of=material] (dynamik) {\bf Unterlagen\_Dynamik};
  % 1st level edges
  \path [line] (parent) |- (dynamik);
  \path [line] (parent) |- (kmech);
  \path [line] (parent) |- (material); %| vline to disable odd look in emacs
  %
  % 2nd level nodes
  \node [obsolete, right of=kmech, node distance=6cm] (unkmech)
   {unwichtig};
  \node [obsolete, right of=material, node distance=6cm] (unmaterial)
   {unwichtig};
  \node [virtualnode, below of=dynamik] (vndyn) {};
  \node [ldir, right of=vndyn, node distance=3.5cm] (formel)
   {\bf Formelsammlung\_Dynamik};
  \node [ldir, below of=formel] (uebung) {\bf \"{U}bungsbl\"{a}tter\_Dynamik} ;
  \node [ldir, below of=uebung] (org) {LV\_Organisation};
  % 2nd level edges
  \path [line] (kmech) |- (unkmech);
  \path [line] (material) |- (unmaterial);
  \path [line] (dynamik) |- (formel);
  \path [line] (dynamik) |- (uebung);
  \path [line] (dynamik) |- (org); %| vline to disable odd look in emacs
  %
  % 3rd level nodes
  \node [description, right of=formel, node distance=5.5cm] (wichtigformel)
   {Genaueres\\ im Folgenden};
  \node [description, right of=uebung, node distance=5.5cm] (wichtiguebung)
   {Genaueres\\ im Folgenden};
  \node [virtualnode, below of=org] (vnorg) {};
  \node [ldir, right of=vnorg, node distance=3.5cm] (archiv)
   {Archiv};
  \node [description, below of=archiv] (wichtigarchiv)
   {Daten der Vorjahres-LVs};
  \node [ldir, below of=wichtigarchiv] (aktuell) {Aktuelle\_Lehrveranstaltung};
  % 3rd level edges
  \path [line] (formel) |- (wichtigformel);
  \path [line] (uebung) |- (wichtiguebung);
  \path [line] (archiv) -- (wichtigarchiv);
  \path [line] (org) |- (archiv);
  \path [line] (org) |- (aktuell); %| vline to disable odd look in emacs
  %
  % 4th level nodes
  \node [virtualnode, below of=aktuell] (vnaktuell) {};
  \node [filemisc, left of=vnaktuell, node distance=3.5cm] (beurteilung)
   {Beurteilungslisten};
  \node [filemisc, below of=beurteilung] (modal) {Pr\"{u}fungsmodalit\"{a}ten};
  \node [filemisc, below of=modal] (plan) {Studenplan};
  \node [filemisc, below of=plan] (teilnehmer) {Teilnehmerlisten};
  \node [filemisc, below of=plan] (abgabe) {\"{U}bungsblattabgabe};
  % 4th level edges
  \path [line] (aktuell) |- (beurteilung);
  \path [line] (aktuell) |- (modal);
  \path [line] (aktuell) |- (plan);
  \path [line] (aktuell) |- (teilnehmer);
  \path [line] (aktuell) |- (abgabe); %| vline to disable odd look in emacs
\end{tikzpicture}
\caption{Übersicht zum Ordner \twrite{Lehrveranstaltungen}.}
\label{fig:lvs}
\end{figure}

\section{Ordner \twrite{Formelsammlung\_Dynamik}}

Wie in \figref{fig:formelsammlung} dargelegt, befinden sich im Unterordner 
\twrite{Formelsammlung\_Dynamik} alle Versionen der Formelsammlung der 
vergangenen Jahre. Wie bereits erwähnt, wurde die \glqq{}Formelsammlung\grqq{} 
im Jahr 2013 auf \glqq{}Vorlesungsbehelf\grqq{} umgetauft. Grundsätzlich ist 
aber nur der Ordner mit dem aktuellen Studienjahr relevant, welches im 
Folgenden als \twrite{yyyy\_zzzz} bezeichnet wird. Somit ist der Ordner 
\twrite{Vorlesungsbehelf\_yyyy\_zzzz} der einzig relevante, da alle anderen nur
der Archivierung dienen.

Am Ende des Sommersemesters wird der Ordner
\twrite{Vorlesungsbehelf\_yyyy\_zzzz} kopiert und mit den Jahreszahlen des
kommenden Semester ausgestattet, und nur noch dieser neue Ordner bearbeitet. 
Somit wird sichergestellt, dass immer eine funktionsfähige Version der 
Formelsammlung des Vorjahres vorhanden ist.

% tikzpic for [Formelsammlung_Dynamik] 
\begin{figure}
\begin{tikzpicture}[node distance = 1.5cm, auto]
  % 1st level nodes
  \node [root] (parent) {Formelsammlung\_Dynamik};
  \node [virtualnode, below of=parent] (vn) {};
  \node [filemisc, right of=vn, node distance = 4.5cm] (fsalt) 
   {Alte\\ Formelsammlungen};
  \node [ldir, below of=fsalt] (fs13) {Vorlesungsbehelf\_2013\_2014};
  \node [ldir, below of=fs13] (fs14) {Vorlesungsbehelf\_2014\_2015};
  \node [ldir, below of=fs14] (fs15) {Vorlesungsbehelf\_2015\_2016};
  \node [ldir, below of=fs15] (fsaktuell) {\bf Vorlesungsbehelf\_yyyy\_zzzz};
  % 1st level edges
  \path [line] (parent) |- (fsalt);
  \path [line] (parent) |- (fs13);
  \path [line] (parent) |- (fs14);
  \path [line] (parent) |- (fs15);
  \path [line] (parent) |- (fsaktuell); %| vline to disable odd look in emacs
  %
  % 2nd level nodes
  \node [description, right of=fsaktuell, node distance=5.5cm] (wichtigakt)
   {Aktuelles Studienjahr};
  % 2st level edges
  \path [line] (fsaktuell) |- (wichtigakt);%| vline to disable odd look in emacs
\end{tikzpicture}
\caption{Übersicht zum Ordner \twrite{Formelsammlung\_Dynamik}.}
\label{fig:formelsammlung}
\end{figure}


\subsection{Unterordner \twrite{Vorlesungsbehelf\_yyyy\_zzzz}}
In \figref{fig:fsakt} wird der Aufbau der Ordnerstruktur der aktuellen
Formelsammlung dargestellt. Wie wichtigsten Ordner und Dateien sind:

\begin{itemize}
  \item \twrite{bilder\_eps:} Alle Bilder, die in der Formelsammlung eingebunden
    werden, müssen im Format {\tt eps} vorliegen. Im Allgemeinen werden die
    Bilder in {\tt Corel-Draw} gezeichnet, und als {\tt eps} exportiert. Dieser
    Ordner, dient zur Ablage aller {\tt eps}-Dateien, damit nur Pfad angegeben
    werden muss, um auf alle Grafiken zugreifen zu können.
  \item \twrite{Bilder\_Corel\_Draw:} Hier befinden sich alle Grafiken, die
    in der Formelsammlung zu Anwendung kommen im Format {\tt crd}. 
    Grundsätzlich sind die Dateien in Unterordnern zu den einzelnen Kapiteln
    angeordnet. Es sind zwar auch einige files außerhalb der Kapitel zu finden,
    jedoch werden diese in der Formelsammlung selbst nie gebraucht.
  \item \twrite{content:} In diesem Ordner sind alle {\tt tex}-files der
    einzelnen Kapitel und des Deckblattes zu finden. Sollten Änderungen
    am Inhalt der Formelsammlung vorgenommen werden, so findet man die meisten
    notwendigen Dateien hier.
  \item \twrite{style:} Hier befinden sich die Dateien {\tt zs\_preambel.tex}
    und {\tt macros.tex}. Das \glqq{}zs\grqq{} bezieht sich hier auf 
    \glqq{}zweiseitig\grqq{}. Es gibt zwar hier auch ein {\tt es\_preambel.tex}
    (\glqq{}einseitig\grqq{}), jedoch wird die Formelsammlung seit 2013 
    eigentlich nur mehr zweiseitig ins TC gestellt. Grundsätzlich sollten diese
    Dateien {\bf nicht} geändert werden (außer man weiß genau, was man tut)!
    Wofür diese Dateien zuständig sind, wird im nächsten Kapitel skizziert.
  \item \twrite{Vorlesungsbehelf\_Mechanik\_B3.tex:} Das \LaTeX-file zur
    Erstellung der Formelsammlung. Darin werden das Deckblatt und die Kapitel
    kompiliert und als \twrite{Vorlesungsbehelf\_Mechanik\_B3.pdf}
    exportiert. Die genauere Funktionsweise wird im nächsten Kapitel erklärt.
\end{itemize}

% tikzpic for [Vorlesungsbehelf_yyyy_zzzz] 
\begin{figure}
\begin{tikzpicture}[node distance = 1.5cm, auto]
  % 1st level nodes
  \node [root] (parent) {Vorlesungsbehelf\_yyyy\_zzzz};
  \node [description, right of=parent, node distance=5cm] (wichtigpar)
   {Aktuelles Studienjahr};
  \node [virtualnode, below of=parent] (vn) {};
  \node [sdir, right of=vn, node distance = 4cm] (eps) {\bf bilder\_eps};
  \node [sdir, below of=eps] (cd) {\bf Bilder\_Corel\_Draw};
  \node [sdir, below of=cd] (content) {\bf content};
  \node [sdir, below of=content] (schwingg) {Erg\"{a}nzung Schwingg};
  \node [sdir, below of=schwingg] (img) {img};
  \node [sdir, below of=img] (junk) {Junk};
  \node [sdir, below of=junk] (print) {print};
  \node [sdir, below of=print] (style) {\bf style};
  \node [lpdf, below of=style] (fspdf) {Vorlesungsbehelf\_Mechanik\_B3.pdf};
  \node [ltex, below of=fspdf] (fstex) {Vorlesungsbehelf\_Mechanik\_B3.tex};
  % 1st level edges
  \path [line] (parent) |- (wichtigpar);
  \path [line] (parent) |- (eps);
  \path [line] (parent) |- (cd);
  \path [line] (parent) |- (content);
  \path [line] (parent) |- (schwingg);
  \path [line] (parent) |- (img);
  \path [line] (parent) |- (junk);
  \path [line] (parent) |- (print);
  \path [line] (parent) |- (style);
  \path [line] (parent) |- (fspdf);
  \path [line] (parent) |- (fstex);%| vline to disable odd look in emacs
  %
  % 2nd level nodes
  \node [description, right of=eps, node distance=5cm] (wichtigeps)
   {Sammlung aller Bilder in eps};
  \node [description, right of=cd, node distance=5cm] (wichtigcd)
   {In Kapitel gegliederte crd-files};
  \node [description, right of=content, node distance=5cm] (wichtigcontent)
   {.tex files (Kapitel, Deckblatt)};
  \node [obsolete, right of=schwingg, node distance=5cm] (unschwingg)
   {obsolet};
  \node [obsolete, right of=img, node distance=5cm] (unimg)
   {unwichtig};
  \node [obsolete, right of=junk, node distance=5cm] (unjunk)
   {unwichtig};
  \node [obsolete, right of=print, node distance=5cm] (unprint)
   {unwichtig};
  \node [description, right of=style, node distance=5cm] (wichtigstyle)
   {zs\_preambel.tex, macro.tex};
  % 2nd level edges
  \path [line] (eps) |- (wichtigeps);
  \path [line] (cd) |- (wichtigcd);
  \path [line] (content) |- (wichtigcontent);
  \path [line] (schwingg) |- (unschwingg);
  \path [line] (img) |- (unimg);
  \path [line] (junk) |- (unjunk);
  \path [line] (print) |- (unprint);
  \path [line] (style) |- (wichtigstyle);%| vline to disable odd look in emacs
\end{tikzpicture}
\caption{Übersicht zum Ordner der aktuellen Formelsammlung.}
\label{fig:fsakt}
\end{figure}

\section{Ordner \twrite{\ddot{U}bungsbl\ddot{a}tter\_Dynamik}}

Wie in \figref{fig:ueblaett} gut erkennbar ist, werden hier alle Daten für
die Übungsaufgaben, die Lösungen dazu und alle durchgerechneten Beispiele
verwaltet. Im Gegensatz zur Formelsammlung gibt es keine Unterordner der
einzelnen Studienjahre, es werden ledigilich die {\tt .pdf}-Dokumente der 
Vorjahre archiviert (aber nicht die gesamte Masse an {\tt .tex}, {\tt .eps} und 
{\tt .crd} Dateien). Die wichtigsten Ordner und Dateien werden nun abermals kurz
kommentiert.

\newpage
\begin{itemize}
  \item \twrite{Aktuelle\_Lehrveranstaltung:} Dieser Ordner beinhaltet im
    Wesentlichen die drei Dateien \twrite{Aufgaben\_WSyyzz.tex},
    \twrite{Durchgerechnet\_WSyyzz.tex} und 
    \twrite{L\ddot{o}sungen\_WSyyzz.tex}. Diese Dateien bestimmen, welche
    Aufgaben / durchgerechnete Beispiele / Lösungen in den finalen
    {\tt .pdf}-Dokumenten aufscheinen. Am Ende des Sommersemester werden diese
    drei files kopiert und mit den Jahreszahlen des kommenden Studienjahres
    ausgestattet. Die alten Dateien können in den Ordner
    \twrite{Archiv~Studienunterlagen} (bzw. in einen Unterordner mit den
    entsprechenden Jahreszahlen) verschoben werden. Die Funktionsweise der drei
    \LaTeX-files wird im nächsten Kapitel genau beschrieben.
  \item \twrite{Archiv Studienunterlagen:} Ablage alter Dateien der Vorjahre.
  \item \twrite{Basisdaten:} Hier befinden sich alle wichtigen Dateien, wie
    beispielsweise die Deckblätter und die einzelnen Beispiele.
  \item \twrite{Aufgaben.tex:} Diese \LaTeX-file dient zur Erstelllung
    {\bf aller drei} übungsbezogenen Dokumente (d.h. Angaben, Durchgerechnete
    {\bf und} Lösungen). Der Inhalt wird in \twrite{Aufgaben.pdf} exportiert
    und die genaue Funktion im zweiten Kapitel erklärt.
  \item \twrite{Basisdaten/crd:} Alle {\tt .crd}-files. Es gibt hier keine
    Unterteilung in Kapitel oder Unterscheidung zwischen Angaben,
    Durchgerechnete oder Lösungen.
  \item \twrite{Basisdaten/eps:} Alle {\tt .eps}-files. Es gibt hier keine
    Unterteilung in Kapitel oder Unterscheidung zwischen Angaben,
    Durchgerechnete oder Lösungen.
  \item \twrite{Basisdaten/style:} Hier befinden sich neben der 
    \twrite{preambel.tex} (an der man abermals nichts ändern sollte), die
    Datei \twrite{variable.tex} und die Deckblätter. In \twrite{variable.tex}
    befinden sich, wie der Name schon suggeriert, viele Variablen, die beim
    Übergang von ein Studienjahr auf das nächste geändert werden müssen. Diese
    sind zum Beispiel die Nummer der für die Übungsblattabgabe relevanten
    Beispiele, die Termine der Prüfungen und das Studienjahr selbst. Die
    Handhabung von \twrite{variable.tex} und der Deckblätter wird im nächsten
    Kapitel noch weiter vertieft.
  \item \twrite{Basisdaten/tex:} Hier befinden sich die \LaTeX-files zu den
    eigentlichen Beispielen. Beispielsweise befinden sich im Unterordner
    \twrite{Basisdaten/tex/Aufgaben} die files \\\twrite{Aufgabe\_01.tex},
    \twrite{Aufgabe\_02.tex},\dots,\twrite{Aufgabe\_nn.tex},d.h. jede Aufgabe
    hat hier ihr eigenes {\tt .tex}-file. Müssen Änderungen an einzelnen
    Beispielen vorgenommen werden, so können also hier die notwendigen Dateien
    gefunden werden. Die beiden anderen Unterordner für die Durchgerechneten 
    und die Lösungen sind gleich aufgebaut.

    {\bf Wichtig:} Die Nummerierung der Beispiele ist mehr oder weniger
    willkürlich. Nur weil eine Aufgabe die Nummer 02 hat (also 
    \twrite{Aufgabe\_02.tex}), muss diese nicht notwendigerweise als zweite
    Aufgabe in den Angaben aufscheinen. Die Nummerierung hier erfolgt eigentlich
    nach dem Zeitpunkt des Erstellens eines Beispiels. Wird also ein neues
    Beispiel erstellt, so sollte dieses Beispiel die Nummer des kleinstmöglichen
    noch freien Indexes bekommen.\\
    Zur Nummerierung der Beispiele wird noch etwas detaillierter im zweiten
    Kapitel Stellung genommen.
\end{itemize}

% tikzpic for [Uebungsblaetter_Dynamik]
\begin{figure} 
\begin{tikzpicture}[node distance = 1.4cm, auto]
  % 1st level nodes
  \node [root] (parent) {\"{U}bungsbl\"{a}tter\_Dynamik};
  \node [virtualnode, below of=parent] (vn) {};
  \node [ldir, right of=vn, node distance = 3.5cm] (aktuell)
   {\bf Aktuelle\_Lehrveranstaltung};
  \node [ldir, below of=aktuell] (archiv) {Archiv Studienunterlagen};
  \node [ldir, below of=archiv] (basis) {\bf Basisdaten};
  \node [spdf, below of=basis] (apdf) {Aufgaben.pdf};
  \node [stex, below of=apdf] (atex) {Aufgaben.tex};
  % 1st level edges
  \path [line] (parent) |- (aktuell);
  \path [line] (parent) |- (archiv);
  \path [line] (parent) |- (basis);
  \path [line] (parent) |- (apdf);
  \path [line] (parent) |- (atex);%| vline to disable odd look in emacs
  %
  % 2nd level nodes
  \node [virtualnode, right of=aktuell, node distance=3.5cm] (vnxx) {};
  \node [ltex, right of=vnxx, node distance=3.5cm] (lxx)
   {L\"{o}sungen\_WSyyzz.tex};
  \node [ltex, above of=lxx] (dxx) {Durchgerechnet\_WSyyzz.tex};
  \node [ltex, above of=dxx] (axx) {Aufgaben\_WSyyzz.tex};
  \node [description, right of=archiv, node distance=7cm] (wichtigarchiv)
   {.pdf-Dateien der Vorjahres-LVs};
  \node [virtualnode, right of=basis, node distance=9cm] (vnbase) {};
  \node [virtualnode, below of=vnbase, node distance=5cm] (vnbranch) {};
  \node [sdir, left of=vnbranch, node distance=3cm] (crd) {\bf crd};
  \node [sdir, below of=crd] (eps) {\bf eps};
  \node [sdir, below of=eps] (style) {\bf style};
  \node [sdir, below of=style] (tex) {\bf tex};
  % 2nd level edges
  \path [line] (aktuell) |- (lxx);
  \path [line] (vnxx) |- (dxx);
  \path [line] (vnxx) |- (axx);
  \path [line] (archiv) |- (wichtigarchiv);
  \path [line] (basis) -- (vnbase);
  \path [line] (vnbase) |- (crd);
  \path [line] (vnbase) |- (eps);
  \path [line] (vnbase) |- (style);
  \path [line] (vnbase) |- (tex);%| vline to disable odd look in emacs
  %
  % 3rd level nodes
  \node [description, left of=crd, node distance=5cm] (wichtigcrd)
   {Alle .crd files};
  \node [description, left of=eps, node distance=5cm] (wichtigeps)
   {Alle .eps files};
  \node [virtualnode, left of=style, node distance=4cm] (vnstyle) {};
  \node [stex, left of=vnstyle, node distance=5cm] (pretex) {preambel.tex};
  \node [stex, below of=pretex] (vartex) {variable.tex};
  \node [ltex, below of=vartex] (adb) {Aufgaben\_Deckblatt.tex};
  \node [ltex, below of=adb] (ddb) {Durchgerechnet\_Deckblatt.tex};
  \node [ltex, below of=ddb] (ldb) {L\"{o}sungen\_Deckblatt.tex};
  \node [sdir, below of=tex, node distance=5.5cm, shift=(left:2.5cm)] (adir)
   {\bf Aufgaben};
  \node [sdir, below of=adir] (ddir) {\bf Durchgerechnet};
  \node [sdir, below of=ddir] (ldir) {\bf L\"{o}sungen};
  \node [ldescription, left of=ddir, node distance=6cm] (wichtigadl)
   {In diesen Ordnern hat jedes Beispiel ein eigenes .tex file:\\
     Aufgabe\_xx.tex\\Durchgerechnet\_xx.tex\\L\"{o}sung\_xx.tex};
  % 3rd level edges
  \path [line] (crd) -- (wichtigcrd);
  \path [line] (eps) -- (wichtigeps);
  \path [line] (style) -- (pretex);
  \path [line] (vnstyle) |- (vartex);
  \path [line] (vnstyle) |- (adb);
  \path [line] (vnstyle) |- (ddb);
  \path [line] (vnstyle) |- (ldb);
  \path [line] (tex) |- (adir);
  \path [line] (tex) |- (ddir);
  \path [line] (tex) |- (ldir);
  \path [line] (adir) -| (wichtigadl);
  \path [line] (ddir) -- (wichtigadl);
  \path [line] (ldir) -| (wichtigadl);%| vline to disable odd look in emacs
\end{tikzpicture}
\caption{Übersicht zum Ordner \twrite{\ddot{U}bungsbl\ddot{a}tter\_Dynamik}.}
\label{fig:ueblaett}
\end{figure}


%\begin{landscape}
%Test.
%\end{landscape}